\documentclass{article}
\usepackage{amsmath} % For math symbols and environments
\usepackage{graphicx} % Required for including images
\usepackage{amssymb} % For checkmark symbol if needed, and other symbols
\usepackage{makecell} % For multi-line table cells
\usepackage{tikz}
\usepackage{pgfplots}
\pgfplotsset{compat=1.17}
\usepackage{standalone}
\usepackage{booktabs}
\usepackage{parskip} % Add space between paragraphs
\usetikzlibrary{positioning,arrows.meta,calc,decorations.pathmorphing,fit,shapes.geometric}
\setlength{\parindent}{0pt} % Remove paragraph indentation

\title{Lab Report 5: Optical Qubits and Quantum Interference}

\author{
  Vincent Migliaccio \and
  Anthony Martinez \and
  Adrian Rios \and
  Paul Gauthier
  \thanks{Group Number: 13} \\
}

\date{\today}

\begin{document}
\pagestyle{empty}

% Custom style for quantum optics
\tikzset{
    >={Straight Barb[width=3pt,length=3pt]},  % Narrower default arrow
    photon/.style={thick, red},
    detector/.style={
        draw=black, 
        semicircle,
        minimum width=0.6cm,
        minimum height=0.6cm,
        inner sep=0pt,
        outer sep=0pt,
        fill=yellow!10, 
        text=black
    },
    hwp/.style={draw, rectangle, minimum width=0.25cm, minimum height=0.75cm, fill=blue!10, thick},
    lp/.style={draw, rectangle, minimum width=0.25cm, minimum height=0.75cm, fill=gray!10, thick},
    box/.style={draw, rectangle, rounded corners, inner sep=6pt},
    beamsplitter/.style={
        minimum width=0.5cm,
        minimum height=0.5cm,
        path picture={
            \draw[very thick,dashed] (path picture bounding box.north east) -- (path picture bounding box.south west);
        }
    },
    pbs/.style={draw, rectangle, minimum width=1cm, minimum height=1cm, fill=gray!20, thick},
    mirror/.style={
        minimum width=0.5cm,
        minimum height=0.5cm,
        path picture={
            \draw[very thick] (path picture bounding box.north east) -- (path picture bounding box.south west);
        }
    },
    stage/.style={
        minimum width=0.5cm,
        minimum height=0.5cm,
        path picture={
            \draw[<->,very thick] (path picture bounding box.north west) -- (path picture bounding box.south east);
        }
    },
}

\begin{figure}[htbp]
\centering
\resizebox{\linewidth}{!}{%
\begin{tikzpicture}[scale=1.8]
    
    % Source
  \node[draw, circle, minimum size=0.5cm] (S) at (0,0) {\small S};

  \node[draw, rectangle, minimum width=0.25cm, minimum height=0.75cm, fill=red!10, thick,rotate=0,right=0.75cm of S] (FW) {};
  \node[below=0.25cm of FW] () {\small Filter Wheel};
  
  \node[hwp,rotate=0,right=3.25cm of S] (HWP) {};
  \node[below=0.25cm of HWP] () {\small HWP$_p(22.5^\circ)$};

  
    \node[draw, rectangle, minimum size=0.5cm, right=5cm of S] (BIBO) at (0,0) {\small BiBO};
    % Idler arm: Mach-Zehnder interferometer (MZI)
    \node[beamsplitter, right=1.5cm of BIBO] (BS1) {};
    \node[below=0.4cm of BS1] () {\small BS$_1$};

    \node[mirror, above=2.5cm of BS1] (M1) {};
    \node[mirror, right=2.5cm of BS1] (M2) {};
    \node[stage] at ($(M2)+(0.2cm,-0.2cm)$) (STAGE) {};
    \node at ($(STAGE)+(0.15cm,0.15cm)$) {$\varphi$};
    \node[right=.4cm of STAGE,align=left] {\small Piezo-controlled\\stage};


    \node[beamsplitter, right=2.5cm of M1] (BS2) {};
    \node at ($(BS2)-(0.3,0.3)$) {\small BS$_2$};

    \node[detector,rotate=270] (DI) at ($(BS2.east)+(1.5cm,0)$) {\rotatebox{90}{\small $D_i$}};

    % Signal arm: LP with detector
    \node[lp,rotate=270] at ($(BIBO)+(0,1.75cm)$) (LPs) {};
    \node[left=0.5cm of LPs] () {\small LP$_s(\theta_s)$};
    \node[detector,rotate=0, above=4cm of BIBO] (DS) {\small $D_s$};

    \node[lp,rotate=0,right=1cm of BS2] (MZI_LP) {};
    \node[below=0.1cm of MZI_LP] () {\small LP$_i(90^\circ)$};

     % For Lab 6
     \node[hwp,rotate=0,right=1.25cm of M1] (MZI_HWP) {};
     \node[above=0.1cm of MZI_HWP] () {\small HWP$_t(45^\circ)$};
     \node[hwp,rotate=0,right=1.25cm of BS1] (MZI_HWP2) {};
     \node[below=0.25cm of MZI_HWP2] () {\small HWP$_b(0^\circ)$};

    % Pump and photon paths
    \draw[->, thick, red] (S) -- (BIBO);

    % Signal (upper) photon
    \draw[->, thick, red] (BIBO) -- (DS);

    % Idler photon through the MZI
    \draw[-, thick, red] (BIBO) -- (BS1.center);
    \draw[-, thick, red] (BS1.center) -- (M1.center);
    \draw[-, thick, red] (BS1.center) -- (M2.center);
    \draw[-, thick, red] (M1.center) -- (BS2.center);
    \draw[->, thick, red] (M2.center) -- ($(BS2.north) + (0,.1cm)$);
    \draw[-, thick, red] (BS2.center) -- (BS2.center);
    \draw[->, thick, red] (BS2.center) -- (DI);
    
    % Coincidence Counter and connecting dashed lines
    \node[draw, rectangle] (CC) at ($(DI |- DS)$) {\small CC};
    \draw[-, dashed] (DI.west) -- (CC.south);
    \draw[-, dashed] (DS.east) -- (CC.west);

    % These overwrite the laser lines
    \node[lp,rotate=270,fill=gray!50,minimum height=1.5cm] at ($(BS1)+(-0.2,0.6cm)$) (SHUTTER) {};
    \node[align=right] at ($(SHUTTER)+(-0.4,0.35)$) {\small Shutter\\(retractable)};

  \node[draw, rectangle, minimum width=0.25cm, minimum height=1.5cm, fill=gray!50, thick,rotate=0] at ($(S)+(1.35cm,0.2)$) (PM) {};
  \node[above=0.25cm of PM,align=center] () {\small Power meter\\(retractable)};



\end{tikzpicture}
}% end resizebox
\caption{
Entangled pair quantum eraser experiment, where the signal photon's
polarization carries ``which-way'' information
about the idler's path through the MZI.
Uses the unchanged Lab 6 apparatus with settings:
1. Set the pump HWP$_p$ to $45^\circ$ to generate $|\Phi^+\rangle$ entangled pairs;
2. Vary the signal LP$_s$ angle $\theta_s$ to control the eraser.
Setting $\theta_s = 45^\circ$ erases the ``which-way'' information from the signal,
restoring self-interference of the idler in the MZI.
The idler does not self-interfere 
for $\theta_s \in \{0^\circ, 90^\circ\}$, as the which-way info remains on the signal.
\label{fig:apparatus}
}
\end{figure}


\subsection*{Introduction}

The existing Lab 6 apparatus seems capable of an even more compelling quantum
eraser experiment that uses entangled pairs.
Rather than marking the idler with its own ``which-way'' information, it is possible
to mark the \emph{signal} with the idler's path through the MZI.
Then by manipulating the signal, we can ``erase'' the which-way information
and restore self-interference of the idler in the MZI.

This experiment can be run simply by inputting new HWP and LP angles
into the normal PC control screen.
With no changes to any apparatus on the optical table.

We ensure that only horizontally polarized idlers
can reach the idler detector D$_i$ via the bottom MZI arm,
and only vertical can reach it via the top arm.
To reach D$_i$ via the bottom arm, idlers only encounter
LP$_i$ which is set to $90^\circ$ (H) -- only
H photons entering the bottom arm can pass.
To reach D$_i$ via the top arm, idlers pass through
MZI HWP$_t$ set to $45^\circ$ before LP$_i$ -- only
V photons entering the top arm can pass, as HWP$_t$
will convert them to H.
Photons entering the top arm as H will become V polarized
and fail to pass LP$_i$.

Using $| \Phi^+ \rangle = \frac{1}{\sqrt2}(|HH\rangle + |VV\rangle)$
entangled pairs means that the signal's polarization marks which
path the idler can take through the MZI.
This which-way info on the signal
prevents the idler from self-interfering in the MZI.
Adjusting the signal LP$_s$ to $45^\circ$ erases the which-way information
by projecting the signal's polarization onto D.
This restores self-interference of the idler in the MZI.

\subsection*{Operator for Lab 6}

In the prelab, we created the MZI operator $\hat{Z}'(\vartheta)$
where $\vartheta$ is the MZI HWP$_t$ angle.
Let's compose a polarizer operator for the idler polarizer
$LP_i$ onto $\hat{Z}'$
to create an operator for the full Lab 6 experiment.

\paragraph{Linear polarizer operator}
Here's an operator for a LP at angle $\theta$,
operating on photons in the H=$90^\circ$, V=$0^\circ$ basis as we have been:

$$\hat{P}(\theta) = \left[\begin{matrix}\sin^{2}{\left(\theta \right)} & - \sin{\left(\theta \right)} \cos{\left(\theta \right)}\\- \sin{\left(\theta \right)} \cos{\left(\theta \right)} & \cos^{2}{\left(\theta \right)}\end{matrix}\right]$$


Since the idler LP$_i$ is placed at the horizontal MZI output, 
we need an operator that applies $\hat{P}$
to the polarization component of the state in spatial state $|\psi_b\rangle$,
while leaving the top path $|\psi_t\rangle$ unchanged.

$$  \hat{P'}(\theta) = (|\psi_b\rangle\langle\psi_b| \otimes \hat{P}(\theta)) + (|\psi_t\rangle\langle\psi_t| \otimes \hat{I}) $$

\paragraph{Lab 6 operator for a single photon}
And now we can compose $\hat{P'}$ and $\hat{Z}'$
to create an operator that encompasses all of the Lab 6 apparatus:

$$\hat{E}'(\theta,\vartheta) = \hat{P'}(\theta) \hat{Z}'(\vartheta)$$

We will use it with the MZI HWP$_t$ set to $\vartheta=45^\circ$ and 
the idler LP$_i$ set to $\theta=90^\circ$, so
$$
\hat{E}'_{45,90}\;=\;
\hat{E}'\!\left(\theta=90^\circ,\;\vartheta=45^\circ\right)
$$


\paragraph{Evaluating with single photons}

As desired, sending single horizontal or vertical polarized photons into
this MZI+LP$_i$ apparatus results in constant idler detection probabilities.
Such photons can only travel one of the arms, so there is no self-interference.
\begin{align*}
|\langle \psi_b,H |\hat{E}'_{45,90} |\psi_b, H\rangle|^2 &= \frac{1}{4} \\
|\langle \psi_b,H |\hat{E}'_{45,90} |\psi_b, V\rangle|^2 &= \frac{1}{4} \\
\end{align*}

Sending in diagonally polarized idlers produces a detection
probability that varies with phase delay.
Since diagonal polarization is a superposition of
$\frac{1}{\sqrt2}(|H\rangle+|V\rangle)$,
contributions from both arms can self-interfere.
\begin{align*}
|\langle \psi_b,H |\hat{E}'_{45,90} |\psi_b, D\rangle|^2 &= \frac{\cos{\left(\delta \right)}}{4} + \frac{1}{4}
\end{align*}

\subsection*{Extension to signal–idler pairs}

Each photon spans a four–dimensional Hilbert space
\((2\text{ paths} \otimes 2\text{ polarisations})\)
so a pair occupies
\(\mathcal{H}_{\mathrm{s}}\otimes\mathcal{H}_{\mathrm{i}}\) (dimension 16).
First, let's apply the Lab 6 operator to the idler photon:
\[
\hat{\mathcal{E}}=
\mathbb{I}_{4}\otimes\hat{E}'_{45,90}
\]

Apply the signal LP$_s$ at angle $\theta$ to the signal photon:
\[
\hat{\mathcal{P}}(\theta)=\hat{P}'(\theta)\otimes\mathbb{I}_{4}
\]

Compose them to form the full entangled pair quantum eraser operator:
\[
\hat{\mathcal{O}}(\theta)=\hat{\mathcal{E}}\;\hat{\mathcal{P}}(\theta)
\]

\paragraph{Input state}

We prepare the Bell state with both photons in the \(b\)-path,
noting that the path state has no physical meaning for signal photons:
\[
|\Phi^{+}\rangle=
\frac{\,|\psi_bH\rangle_{\!s}|\psi_bH\rangle_{\!i}
      +|\psi_bV\rangle_{\!s}|\psi_bV\rangle_{\!i}}{\sqrt{2}}.
\]

\paragraph{Propagation}

After the optical elements
\[
|\Psi_{\mathrm{out}}(\theta)\rangle
=\hat{\mathcal{O}}(\theta)\,|\Phi^{+}\rangle.
\]

\paragraph{Coincident detection}
Coincidences are detected when:

(i) the \emph{signal} photon is detected, with no restriction on its path or polarization

(ii) the \emph{idler} photon is found in the \(b\)-path showing that it exited the MZI and passed through LP$_i$ to the idler detector D$_i$, with no polarization restriction.

This is described by the projector
\[
\hat{\Pi}=\mathbb{I}_{4}\;\otimes\;
\bigl(|\psi_b\rangle\langle\psi_b|\otimes\mathbb{I}_{2}\bigr),
\]
so that the coincidence probability is
\[
P(\theta,\delta)=\langle\Psi_{\mathrm{out}}(\theta)|\hat{\Pi}|\Psi_{\mathrm{out}}(\theta)\rangle.
\]

\paragraph{Probabilities}

With the signal LP$_s$ set to $\theta=0^\circ$ or $90^\circ$ the probability of
coincidence detections is constant.
The signal carries which-way information about which path the idler took
through the MZI, so the idler can not self-interfere.
\[
P(0^\circ, \delta)=P(90^\circ, \delta)=\frac{1}{8}
\]

With the signal LP$_s$ set to $\theta=45^\circ$ the probability of
coincidence detections varies with the phase delay $\delta$.
Projecting the signal onto the diagonal polarization erases the
which-way information and restores the idler's self-interference.
\[
P(45^\circ, \delta)=\frac{1-\cos\delta}{8}
\]

\subsection*{Experimental Tasks}

\subsubsection*{Task 1: Setup}

\begin{itemize}
\item Set the pump HWP to $22.5^\circ$ to generate $| \Phi^+ \rangle = \frac{1}{\sqrt2}(|HH\rangle + |VV\rangle)$ pairs.
\item Set the MZI HWP to $45^\circ$ so the upper arm swaps H/V polarizations.
\item Set the MZI/idler LP to $90^\circ$ so only H can pass to the idler detector.
\item Open the MZI arm shutter.
\item Set acquisition time to 10 sec. Long because we're losing $\approx 7/8$ of photons.
\item Set the pump laser power to max.
\item Lower the power meter arm.
\item Record the power meter reading.
\end{itemize}

\begin{table}[h]
\centering
\begin{tabular}{lll}
\toprule
\textbf{Component} & \textbf{Setting} & Notes\\
\midrule
Idler LP  & \textbf{--} & Input is disabled \\
Signal LP & $45^\circ$ & Varies 45, 0, 90 to turn eraser on/off \\
Pump HWP  & $22.5^\circ$ & Generate $|\Phi^+ \rangle$ pairs \\
MZI LP    & $90^\circ$ & Only H photons can reach detector \\
MZI HWP   & $45^\circ$ & Upper arm swaps H/V polarization\\
\bottomrule
\end{tabular}
\end{table}


\subsubsection*{Task 2: Scan with eraser on at $45^\circ$}

\begin{itemize}
\item Set the signal LP to $45^\circ$, to project the signal polarization onto D and erase which-way info.
\item Zero the stage position.
\item Lower the power meter arm, acquire counts, raise power meter arm.  
\item Step through stage positions, acquiring counts at each step until 3 fringes/oscillations in coincidence counts are collected.
\end{itemize}

\subsubsection*{Task 3: Scan with eraser off at $0^\circ$}

\begin{itemize}
\item Set the signal LP to $0^\circ$, so V signals can pass (idler must use top arm).
\item Zero the stage position.
\item Lower the power meter arm, acquire counts, raise power meter arm.  
\item Step through stage positions, acquiring counts at each step. Scan the same range as Task 3.
\end{itemize}

\subsubsection*{Task 4: Scan with eraser off at $90^\circ$}

\begin{itemize}
\item Set the signal LP to $90^\circ$, so H signals can pass (idler must use lower arm).
\item Zero the stage position.
\item Lower the power meter arm, acquire counts, raise power meter arm.  
\item Step through stage positions, acquiring counts at each step. Scan the same range as Task 3.
\end{itemize}

\end{document}  

