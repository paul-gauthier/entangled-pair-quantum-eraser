\documentclass{article}
\usepackage{amsmath} % For math symbols and environments
\usepackage{graphicx} % Required for including images
\usepackage{amssymb} % For checkmark symbol if needed, and other symbols
\usepackage{makecell} % For multi-line table cells
\usepackage{tikz}
\usepackage{pgfplots}
\pgfplotsset{compat=1.17}
\usepackage{standalone}
\usepackage{booktabs}
\usepackage{parskip} % Add space between paragraphs
\usetikzlibrary{positioning,arrows.meta,calc,decorations.pathmorphing,fit,shapes.geometric}
\setlength{\parindent}{0pt} % Remove paragraph indentation

\title{Lab Report 5: Optical Qubits and Quantum Interference}

\author{
  Vincent Migliaccio \and
  Anthony Martinez \and
  Adrian Rios \and
  Paul Gauthier
  \thanks{Group Number: 13} \\
}

\date{\today}

\begin{document}


% Custom style for quantum optics
\tikzset{
    >={Straight Barb[width=3pt,length=3pt]},  % Narrower default arrow
    photon/.style={thick, red},
    detector/.style={
        draw=black, 
        semicircle,
        minimum width=0.6cm,
        minimum height=0.6cm,
        inner sep=0pt,
        outer sep=0pt,
        fill=yellow!10, 
        text=black
    },
    hwp/.style={draw, rectangle, minimum width=0.25cm, minimum height=0.75cm, fill=blue!10, thick},
    lp/.style={draw, rectangle, minimum width=0.25cm, minimum height=0.75cm, fill=gray!10, thick},
    box/.style={draw, rectangle, rounded corners, inner sep=6pt},
    beamsplitter/.style={
        minimum width=0.5cm,
        minimum height=0.5cm,
        path picture={
            \draw[very thick,dashed] (path picture bounding box.north east) -- (path picture bounding box.south west);
        }
    },
    pbs/.style={draw, rectangle, minimum width=1cm, minimum height=1cm, fill=gray!20, thick},
    mirror/.style={
        minimum width=0.5cm,
        minimum height=0.5cm,
        path picture={
            \draw[very thick] (path picture bounding box.north east) -- (path picture bounding box.south west);
        }
    },
    stage/.style={
        minimum width=0.5cm,
        minimum height=0.5cm,
        path picture={
            \draw[<->,very thick] (path picture bounding box.north west) -- (path picture bounding box.south east);
        }
    },
}

\begin{figure}[htbp]
\centering
\resizebox{\linewidth}{!}{%
\begin{tikzpicture}[scale=1.8]
    
    % Source
  \node[draw, circle, minimum size=0.5cm] (S) at (0,0) {\small S};

  \node[draw, rectangle, minimum width=0.25cm, minimum height=0.75cm, fill=red!10, thick,rotate=0,right=0.75cm of S] (FW) {};
  \node[below=0.25cm of FW] () {\small Filter Wheel};
  
  \node[hwp,rotate=0,right=3.25cm of S] (HWP) {};
  \node[below=0.25cm of HWP] () {\small HWP$_p(22.5^\circ)$};

  
    \node[draw, rectangle, minimum size=0.5cm, right=5cm of S] (BIBO) at (0,0) {\small BiBO};
    % Idler arm: Mach-Zehnder interferometer (MZI)
    \node[beamsplitter, right=1.5cm of BIBO] (BS1) {};
    \node[below=0.4cm of BS1] () {\small BS$_1$};

    \node[mirror, above=2.5cm of BS1] (M1) {};
    \node[mirror, right=2.5cm of BS1] (M2) {};
    \node[stage] at ($(M2)+(0.2cm,-0.2cm)$) (STAGE) {};
    \node at ($(STAGE)+(0.15cm,0.15cm)$) {$\varphi$};
    \node[right=.4cm of STAGE,align=left] {\small Piezo-controlled\\stage};


    \node[beamsplitter, right=2.5cm of M1] (BS2) {};
    \node at ($(BS2)-(0.3,0.3)$) {\small BS$_2$};

    \node[detector,rotate=270] (DI) at ($(BS2.east)+(1.5cm,0)$) {\rotatebox{90}{\small $D_i$}};

    % Signal arm: LP with detector
    \node[lp,rotate=270] at ($(BIBO)+(0,1.75cm)$) (LPs) {};
    \node[left=0.5cm of LPs] () {\small LP$_s(\theta_s)$};
    \node[detector,rotate=0, above=4cm of BIBO] (DS) {\small $D_s$};

    \node[lp,rotate=0,right=1cm of BS2] (MZI_LP) {};
    \node[below=0.1cm of MZI_LP] () {\small LP$_i(90^\circ)$};

     % For Lab 6
     \node[hwp,rotate=0,right=1.25cm of M1] (MZI_HWP) {};
     \node[above=0.1cm of MZI_HWP] () {\small HWP$_t(45^\circ)$};
     \node[hwp,rotate=0,right=1.25cm of BS1] (MZI_HWP2) {};
     \node[below=0.25cm of MZI_HWP2] () {\small HWP$_b(0^\circ)$};

    % Pump and photon paths
    \draw[->, thick, red] (S) -- (BIBO);

    % Signal (upper) photon
    \draw[->, thick, red] (BIBO) -- (DS);

    % Idler photon through the MZI
    \draw[-, thick, red] (BIBO) -- (BS1.center);
    \draw[-, thick, red] (BS1.center) -- (M1.center);
    \draw[-, thick, red] (BS1.center) -- (M2.center);
    \draw[-, thick, red] (M1.center) -- (BS2.center);
    \draw[->, thick, red] (M2.center) -- ($(BS2.north) + (0,.1cm)$);
    \draw[-, thick, red] (BS2.center) -- (BS2.center);
    \draw[->, thick, red] (BS2.center) -- (DI);
    
    % Coincidence Counter and connecting dashed lines
    \node[draw, rectangle] (CC) at ($(DI |- DS)$) {\small CC};
    \draw[-, dashed] (DI.west) -- (CC.south);
    \draw[-, dashed] (DS.east) -- (CC.west);

    % These overwrite the laser lines
    \node[lp,rotate=270,fill=gray!50,minimum height=1.5cm] at ($(BS1)+(-0.2,0.6cm)$) (SHUTTER) {};
    \node[align=right] at ($(SHUTTER)+(-0.4,0.35)$) {\small Shutter\\(retractable)};

  \node[draw, rectangle, minimum width=0.25cm, minimum height=1.5cm, fill=gray!50, thick,rotate=0] at ($(S)+(1.35cm,0.2)$) (PM) {};
  \node[above=0.25cm of PM,align=center] () {\small Power meter\\(retractable)};



\end{tikzpicture}
}% end resizebox
\caption{
Entangled pair quantum eraser experiment, where the signal photon's
polarization carries ``which-way'' information
about the idler's path through the MZI.
Uses the unchanged Lab 6 apparatus with settings:
1. Set the pump HWP$_p$ to $45^\circ$ to generate $|\Phi^+\rangle$ entangled pairs;
2. Vary the signal LP$_s$ angle $\theta_s$ to control the eraser.
Setting $\theta_s = 45^\circ$ erases the ``which-way'' information from the signal,
restoring self-interference of the idler in the MZI.
The idler does not self-interfere 
for $\theta_s \in \{0^\circ, 90^\circ\}$, as the which-way info remains on the signal.
\label{fig:apparatus}
}
\end{figure}


\subsection*{Lab 6's Quantum Eraser}

In the prelab, we created the MZI operator $\hat{Z}'(\vartheta)$
Let's compose a polarizer operator onto our MZI operator
to create a quantum eraser operator for the full Lab 6 experiment.

Here's an operator for a LP at angle $\theta$,
operating on photons in the H=$90^\circ$, V=$0^\circ$ basis as we have been:

$$\hat{P}(\theta) = \left[\begin{matrix}\sin^{2}{\left(\theta \right)} & - \sin{\left(\theta \right)} \cos{\left(\theta \right)}\\- \sin{\left(\theta \right)} \cos{\left(\theta \right)} & \cos^{2}{\left(\theta \right)}\end{matrix}\right]$$


Since the MZI LP is placed at the horizontal output, 
we need an operator that applies the polarization filter $\hat{P}$
to the polarization component of the state in spatial state $|\psi_b\rangle$,
while leaving the polarization component in the top path $|\psi_t\rangle$ unchanged.
We can construct an operator $\hat{P'}(\theta)$
using projectors onto the spatial basis states:

$$  \hat{P'}(\theta) = (|\psi_b\rangle\langle\psi_b| \otimes \hat{P}(\theta)) + (|\psi_t\rangle\langle\psi_t| \otimes \hat{I}) $$

And now we can compose $\hat{P'}(\theta)$ as the final operator to create
the quantum eraser operator from Lab 6.

$$\hat{E}'(\theta,\vartheta) = \hat{P'}(\theta) \hat{Z}'(\vartheta)$$

We will use it with the MZI HWP$_u$ at $45^\circ$ and the
MZI LP$_i$ at $90^\circ$, so
$$
\hat{E}'_{45,90}\;\equiv\;
\hat{E}'\!\left(\theta=90^\circ,\;\vartheta=45^\circ\right)
$$

\subsection*{Extension to signal–idler photon pairs}

Each photon spans a four–dimensional Hilbert space
\((2\text{ paths}\otimes 2\text{ polarisations})\);               
the pair occupies
\(\mathcal{H}_{\mathrm{s}}\otimes\mathcal{H}_{\mathrm{i}}\) (dimension 16).
\[
\hat{\mathcal{E}}=
\mathbb{I}_{4}\otimes\hat{E}'_{45,90},\qquad
\hat{\mathcal{P}}(\theta)=\hat{P}'(\theta)\otimes\mathbb{I}_{4},\qquad
\hat{\mathcal{O}}(\theta)=\hat{\mathcal{E}}\;\hat{\mathcal{P}}(\theta).
\]

\paragraph{Input state}

We prepare the Bell state with both photons in the \(b\)-path:

\[
|\Phi^{+}\rangle=
\frac{\,|\psi_bH\rangle_{\!s}|\psi_bH\rangle_{\!i}
      +|\psi_bV\rangle_{\!s}|\psi_bV\rangle_{\!i}}{\sqrt{2}}.
\]

\paragraph{Propagation}

After the optical elements
\[
|\Psi_{\mathrm{out}}(\theta)\rangle
=\hat{\mathcal{O}}(\theta)\,|\Phi^{+}\rangle.
\]

\paragraph{Coincident detection}

(i) the \emph{signal} photon is detected, with \emph{no} restriction on
its path or polarisation; and

(ii) the \emph{idler} photon is found in the \(b\)-path
(its polarisation is not analysed).

Operationally this is described by the projector
\[
\hat{\Pi}=\mathbb{I}_{4}\;\otimes\;
\bigl(|\psi_b\rangle\langle\psi_b|\otimes\mathbb{I}_{2}\bigr),
\]
so that the coincidence probability is
\(P(\delta)=\langle\Psi_{\mathrm{out}}|\hat{\Pi}|\Psi_{\mathrm{out}}\rangle.\)
\]

\paragraph{Probability and amplitude}

With the signal polariser fixed at \(\theta=\pi/4\) one finds
\[
P(\delta)=\frac{1-\cos\delta}{8},\qquad
\mathcal{A}(\delta)=\sqrt{P(\delta)}
                  =\frac{\bigl|\sin(\delta/2)\bigr|}{2},
\]
in perfect agreement with the expressions computed by
\texttt{sympy}.

\end{document}  
