\documentclass{article}
\usepackage{graphicx} % Required for including images
\usepackage{booktabs} % For professional looking tables
\usepackage{amsmath}  % For math symbols if needed
\usepackage{parskip}  % Add space between paragraphs
\setlength{\parindent}{0pt} % Remove paragraph indentation
\usepackage[margin=1in]{geometry} % Set margins to 1 inch

\title{Tuesday Lab Session Results: Visibilities and Graphs}
\author{Group 13} % Assuming the same group as lab-6-entangled.tex
\date{\today}

\begin{document}
\maketitle
\pagestyle{empty} % No page numbers for a single page doc

\section*{Visibilities Obtained}

The following visibilities were observed during Tuesday's lab session:

\begin{table}[h!]
\centering
\begin{tabular}{lc}
\toprule
\textbf{Condition} & \textbf{Visibility (V)} \\
\midrule
Eraser Off (Signal LP at $0^\circ$ or $90^\circ$) & 0.351 \\
Eraser On (Signal LP at $45^\circ$)          & 0.526 \\
\bottomrule
\end{tabular}
\caption{Observed visibilities for the quantum eraser experiment.}
\end{table}

\section*{Results Graphs}

\subsection*{Eraser Off}

Figure~\ref{fig:eraser_off} shows the counts versus phase delay when the quantum eraser is "off". In this configuration, which-way information is available, and interference is reduced.

\begin{figure}[h!]
\centering
\includegraphics[width=0.9\textwidth]{counts_vs_phase_delay_eraser_off.pdf}
\caption{Counts vs Phase Delay with the eraser off.}
\label{fig:eraser_off}
\end{figure}

\subsection*{Eraser On}

Figure~\ref{fig:eraser_on} shows the counts versus phase delay when the quantum eraser is "on". By projecting the signal photon onto a diagonal basis, which-way information is erased, and interference is restored.

\begin{figure}[h!]
\centering
\includegraphics[width=0.9\textwidth]{counts_vs_phase_delay_eraser_on.pdf}
\caption{Counts vs Phase Delay with the eraser on.}
\label{fig:eraser_on}
\end{figure}

\end{document}

