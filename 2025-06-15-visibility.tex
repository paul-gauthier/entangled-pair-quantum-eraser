\documentclass{article}
\usepackage{graphicx} % Required for including images
\usepackage{booktabs} % For professional looking tables
\usepackage{makecell} % For line breaks in table cells
\usepackage{amsmath}  % For math symbols if needed
\usepackage{parskip}  % Add space between paragraphs
\setlength{\parindent}{0pt} % Remove paragraph indentation
\usepackage[margin=0.75in]{geometry} % Set margins to 1 inch
\usepackage{caption}  % For caption* to suppress figure numbering
\usepackage{tikz}
\usepackage{pgfplots}
\pgfplotsset{compat=1.17}
\usepackage{standalone}
\usepackage{verbatim}
\usetikzlibrary{positioning,arrows.meta,calc,decorations.pathmorphing,fit,shapes.geometric}

\title{Quantum Eraser Experiment Results: Visibilities and Graphs (2025-06-15)}
\author{Group 13} % Assuming the same group as lab-6-entangled.tex
\date{\today}

\newcommand{\figwidth}{0.64\textwidth}

\begin{document}
\pagestyle{empty} % No page numbers for a single page doc

\begin{table*}[t]
\centering
\begin{tabular}{lccccc}
\toprule
Condition & \makecell{Signal \\ LP angle} & \makecell{Coincidence \\ visibility \\ ($V_{\text{c}}$)} & \makecell{Idler Singles \\ visibility \\ ($V_{\text{i}}$)} & \makecell{Coincidence \\ fringe amplitude \\ ($A_{\text{c}}$)} & \makecell{Idler Singles \\ fringe amplitude \\ ($A_{\text{i}}$)} \\
\midrule
Eraser On   & $45^\circ$ & $0.3435 \pm 0.0031$ & $0.0056 \pm 0.0007$ & $596 \pm 7$ & $470 \pm 55$ \\
Eraser Off  & $0^\circ$  & $0.0837 \pm 0.0040$ & $0.0054 \pm 0.0007$ & $153 \pm 8$ & $452 \pm 59$ \\
%Signal Photons Blocked & N/A & N/A & $0.0047 \pm 0.0005$ & N/A & $391.8 \pm 40.6$ \\
\bottomrule
\end{tabular}
\caption*{
  Turning on the eraser
  quadruples the coincidence interference fringe \emph{amplitude},
  from
   $A_{\text{c,off}} = 153 \pm 8$ to
  $A_{\text{c,on}} = 596 \pm 7$. 
  The idler singles amplitude remains statistically unchanged 
  at $A_{\text{i}} \approx 460 \pm 60$, regardless of the eraser setting. 
  This confirms that erasing is a \emph{post-selection} effect,
  only impacting
  the coincidence counts.
  Results are based on photon counts from 
  312 acquisitions of 30-sec,  
  for both eraser-on and off.
}
\end{table*}

\begin{figure}[h!]
\centering
\includegraphics[width=\figwidth]{data/2025-06-15-09-31-29--sigaligned-on_joint.pdf}
\caption*{
  \textbf{Eraser On} ($\text{signal LP}=45^\circ$).
  Erasing the which-way information increases the coincidence fringe amplitude to
  $A_{\text{c,on}} = 596 \pm 7$.
  The idler singles remain unaffected at
  $A_{\text{i,on}} = 470 \pm 55$.
}
\end{figure}

\begin{figure}[h!]
\centering
\includegraphics[width=\figwidth]{data/2025-06-15-09-31-29--sigaligned-off_joint.pdf}
\caption*{
  \textbf{Eraser Off} ($\text{signal LP}=0^\circ$).  With which-way information left
  intact, the coincidence fringe drops to
  $A_{\text{c,off}} = 153 \pm 8$.
  The idler singles amplitude remains the same within uncertainties at
  $A_{\text{i,off}} = 452 \pm 59$.
}
\end{figure}

\emph{Note: All four graph panels use the same y-axis scale, to facilitate visual comparisons.}

\begin{comment}
\begin{figure}[h!]
\centering
\includegraphics[width=0.50\textwidth]{data/2025-06-15-09-31-29--sigaligned-blocked_joint.pdf}
\caption*{
  With one path blocked, no interference is expected, and coincidence visibility is near zero ($V_{\text{c,blocked}}=0.1887 \pm 1.6822$). The idler singles visibility is also low ($V_{\text{i,blocked}}=0.0047 \pm 0.0005$).
}
\end{figure}
\end{comment}

\end{document}
